\documentclass{article}
%\usepackage{/HAStuSemBonnSuTe25/mysty}
%\usepackage{/HAStuSemBonnSuTe25/my_usual-pack}
%\usepackage{/HAStuSemBonnSuTe25/my_abbr}
\usepackage{my_usual-pack}
\usepackage{my_abbr}
\usepackage{my_letters}


\usepackage[sorting=none,autolang=other,backend=biber,
	bibstyle=alphabetic,citestyle=alphabetic,
	]{biblatex}
\addbibresource{refs.bib}

\title{\vspace{-60pt}Reading Group: Higher Algebra}
\author{Jonas Heintze\thanks{\href{mailto:s6johei2@uni-bonn.de}{
      \url{s6johei2@uni-bonn.de}}}~~and 
  Yordan Toshev\thanks{\href{mailto:yordan.toshev@uni-bonn.de}{
\url{yordan.toshev@gmail.com}}}}
\date{Summer Term 2025}

\begin{document}
\maketitle
%\begin{center}
%%\textbf{\Large{Higher Algebra}}
%  \huge{\bf{Higher Algebra}}\vspace{5pt}
%\\
%\textbf{Seminar Proposal}\\
%Jonas Heintze\thanks{\href{mailto:s6johei2@uni-bonn.de}{\url{s6johei2@uni-bonn.de}}}, Yordan %Toshev\thanks{\href{mailto:yordan.toshev@gmail.com}{\url{yordan.toshev@gmail.com}}}\\
%Summer Term 2025
%\end{center}

%\vspace{1cm}

%\setcounter{section}{4}
%\section*{Sheet 2 - Exercise 4}

\vspace{-5pt}
The seminar aims to cover basics of Higher Algebra, mainly based
upon \cite[Chapter II]{K-ThrNotes}.
The seminar ended last term by defining monoids and 
group objects in an $\infty$-category.
Starting from there, we want to delve into the
theory of stable $\infty$-categories, develop some
of their properties and get some working knowledge.
Among others, we want to examine the process of
stabilisation, define monoidal structures (via $\infty$-operads) 
and localisations 
of stable $\infty$-categories and consider prime examples such as
the category of spectra $\Sp$ and derived category of a 
ring $\cD(R)$.
These constructions appear in many modern approaches across different areas of 
mathematics,
for example
Stable Homotopy Theory ($K$-theory, chromatic homotopy theory),
Arithmetic Geometry ($p$-adic Hodge theory, \dots),
Derived Algebraic Geometry etc.


\section*{Preliminaries}
Knowledge of $\infty$-categories
(the straightening-unstraightening equivalence, (co)limits,
adjunctions, Yoneda's lemma, Kan extensions in the $\infty$-categorical
setting) will be assumed.
A good resource to get on track is \cite[Chapters 1-6]{infty-CatNotes}.
Even if you already know the topics, please briefly revise them,
to make sure we all start on the same page.

\section*{Schedule}
The seminar meets at 18ct on Monday in Room SR TBD.
Each talk discusses roughly 10 pages of \cite{K-ThrNotes} and 
is divided between two speakers, each
preparing a presentation of about 1h.
The precise timing is left to the speakers. Speakers in brackets with (?) are not entirely fixed yet. You can still claim those talks if you want to.
\begin{table}[H]
    \center
    \begin{tabular}{ l l l l}
    \textnumero
    &\bf{Date}\hspace{.5cm}
    &\bf{Topic}
    &\bf{Speakers}\\[.3cm]
    0
    &07.04
    &Recollections on $\infty$-Engineering
    &Jonas,Yordan\\
    1
    &14.04
    &Complete Segal Spaces
    &\\
    2
    &21.04
    &$\bE_1$-Structures
    &\\
    3
    &28.04
    &
    $\bE_\infty$-Structures I
    & Sam\\
    4
    &05.05
    &
    $\bE_\infty$-Structures II and Stable $\infty$-Categories
    &\\
    5
    &12.05
    &
    Stable $\infty$-Categories II
    & Gabriel \\
    6
    &19.05
    &
    $\infty$-Operads I
    &Julius, Carl\\
    7
    &26.05
    &
    $\infty$-Operads II
    &Julius, Carl\\
    8
    &02.06
    &$\infty$-Operads III and Brave New Algebra
    &\\
    -
    &09.06
    &no talk
    &holidays\\
    9
    &16.06
    &
    $\infty$-Categories of Modules
    &\\
    10
    &23.06
    &
    Relative Tensor Products
    &\\
    11
    &30.06
    &
    Localisations of Ring Spectra
    &\\
    12
    &07.07
    &(Open for recommendations)
    &(Melvin(?), Gabriel(?))\\
    13&14.07
    &?&(Melvin(?))\\
\end{tabular}
  \end{table}






\section*{Syllabus}

\subsubsection*{Talk 0: Recollections on $\infty$-Engineering}


\subsubsection*{Talk 1: Complete Segal Spaces}
State Theorem/Definition I.64 and give some intuition and
Explanations I.65 to it.
Work through I.66 to I.69 and sketch some proofs.
Introduce the Quillen Q-construction and talk about I.71.

\subsubsection*{Talk 2: $\bE_1$-Structures}
Motivate (naive definition of) monoids in $\An$ as in II.0 and
give the modern definition as in II.1. State Proposition 2.2 and 
outline the main steps of the proof.
Motivate and state II.3. Summarise II.4 to II.5 with some proof ideas
(if time permits). State II.9, II.10, but focus more on II.11.
Mention briefly (5min) the example from p.71-76.


\subsubsection*{Talk 3: $\bE_\infty$-Structures I}
Motivate and work through II.15. Discuss II.16, II.17 (with very rough outlines
of the proofs). Proof II.18 and make sure to mention II.18a.
Discuss what is relevant from II.18b to be able to state II.21
(feel free to omit the proof). Discuss II.21a (with main focus on parts 
(b) and (c)).

\subsubsection*{Talk 4: $\bE_\infty$-Structures II and 
Stable $\infty$-Categories I}
Finish the discussion on $\bE_\infty$-structures: II.19 and free 
$\bE_\infty$-monoids.
Introduce stable $\infty$-categories and prove II.23. Discuss
homotopy groups, (co)homology of spectra and II.27. Use it to 
tease that $\Sp$ is a stable $\infty$-category.

\subsubsection*{Talk 5: Stable $\infty$-Categories II}
State and prove II.28. Define II.29 and emphasize II.29a.
Discuss II.30 and II.30a and sketch parts of the proofs, which you consider
enlightening. Mention very briefly (max 10min) the ideas of II.31.(a)-(d).
State as a (blackbox) result II.32. State II.33, II.34., II.34a 
(and prove at least one of them if time permits).
Finish the discussion by II.35.

\subsubsection*{Talk 6: $\infty$-Operads I}
Motivate the notion of $\infty$-operads via their $1$-categorical 
precursors and discuss how to give an $\infty$-category a 
(symmetric) monoidal structure via $\infty$-operads. Discuss 
cartesian and cocartesian symmetric monoidal structures of
$\infty$-categories. This entails p.105 - 117.

\subsubsection*{Talk 7: $\infty$-Operads II}
Recall monoids over an $\infty$-operad, in particular II.43 and 
introduce algebras over an $\infty$-operad.
Put emphasis on examples, e.g. II.45a-c.
If time permits, sketch the proof of II.45a.
Discuss Day convolution: II.46 and II.47.
Conclude that $\Sp$ has a \enquote{nice} tensor product.
This entails p.117-124.

\subsubsection*{Talk 8: $\infty$-Operads III and Brave New Algebra}
Present II.48 and II.49. State II.50 and II.51 with emphasis on 
the examples, in particular II.51b. Introduce the tensor product on
spectra, II.52, and use it to discuss II.53. Discuss II.54, II.55 (with proof)
and finish with II.55a.

\subsubsection*{Talk 9: $\infty$-Categories of Modules}
Discuss extensively II.45d-e and less extensively II.56, II.56a,c,e,f,g.
State II.57 (and roughly sketch the proof, if time permits).
State the Schwede-Shipley Theorem, II.58.

\subsubsection*{Talk 10: Tensor Products over Arbitrary Bases}
Go over II.59 and state II.60. Discuss II.61 and II.62.
Mention II.62a-c and sketch II.63. Discuss what you find interesting
from the Miscellanea section.

\subsubsection*{Talk 11: Localisations of $\bE_\infty$-Ring Spectra}
III.1, III.2 (with sketch of proof), III.3, III.4 (and essential parts
of proofs, if times permits), III.4a (discuss why it is not tautological,
c.f. discussion in the proof). Finish with III.5 and preceding example.

\subsubsection*{Talk 12: Open for recommendations }
(Verdier sequences p207-215?, triangulated structures somewhere in chapter IV(?))

\printbibliography
\end{document}
